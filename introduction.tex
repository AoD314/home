\part{О проекте}

\chapter{Как зарожалось}

\chapter{Проектирование}

% Встал вопрос в чем проектировать и как. Есть огромное количество иструментов и еще большее количество, о которых я не ничего не знаю. 

% И так я выбрал отчет в $\LaTeX$ и рачеты в $Python 3$ + $FreeCAD$ для черчения.

\chapter{Зарубежный опыт}

\newcommand{\kwpm}{$\cfrac{\text{кВт·ч}}{\text{м}^2}\ $}

\section{Пассивные дома}
Пассивный дом, энергосберегающий дом, или экодом (нем. Passivhaus, англ. passive house) -- сооружение, основной особенностью которого является отсутствие необходимости отопления или малое энергопотребление - в среднем около 10\% от удельной энергии на единицу объёма, потребляемой большинством современных зданий. В большинстве развитых стран существуют собственные требования к стандарту пассивного дома.

Растут цены на электричество и тепло. Остро стоит вопрос эксплуатационных затрат на жилье. Показателем энергоэффективности объекта служат потери тепловой энергии с квадратного метра \kwpm в год или в отопительный период. В среднем составляет 100—120 \kwpm. Энергосберегающим считается здание, где этот показатель ниже 40 \kwpm.Для европейских стран этот показатель ещё ниже — порядка 10 \kwpm.

Достигается снижение потребления энергии в первую очередь за счет уменьшения теплопотерь здания.

В Европе существует следующая классификация зданий в зависимости от их уровня энергопотребления:
\begin{itemize}
    \item "Старое здание" (здания построенные до 1970-х годов) — они требуют для своего отопления около трехсот киловатт-часов на квадратный метр в год: 300 \kwpm год.
    \item "Новое здание" (которые строились с 1970-х до 2000 года) — не более 150 \kwpm год.
    \item "Дом низкого потребления энергии» (с 2002 года в Европе не разрешено строительство домов более низкого стандарта) — не более 60 \kwpm год.
    \item "Пассивный дом" — не более 15 \kwpm год.
    \item "Дом нулевой энергии" (здание, архитектурно имеющее тот же стандарт, что и пассивный дом, но инженерно оснащенное таким образом, чтобы потреблять исключительно только ту энергию, которую само и вырабатывает) — 0 \kwpm год.
    \item "Дом плюс энергии" или "активный дом" (здание, которое с помощью установленного на нём инженерного оборудования: солнечных батарей, коллекторов, тепловых насосов, рекуператоров, грунтовых теплообменников и т. п. вырабатывало бы больше энергии, чем само потребляло).

\end{itemize}

\section{Энергосберегающие дома}


\section{Экономическая выгода}
	В России энергопотребление в домах составляет 400—600 \kwpm в год.
