\part{Воздух}
\chapter{Немного стандарта}
Стандарт говорит, что воздух нам нужен.

Вопрос в том что непонятно сколько. Возможно достаточно проветривать помещение раз в день - перед сном.

С каждым годом стоимость отопления увеличивается и в погоне за экономией мы стараемся сберечь тепло, которе у нас есть в доме, применяя порой очевидные приемы, которые помогут нам в этом, однако забывая о последствиях. В старых СНиПах 80-х годов было указано, что окна в домах \textbf{должны} обеспечить определенный приток воздуха в помещение. Было дано вполне определенное число по стандарту, сколько иммено воздуха окно должно пропускать. В свою очередь, поступление холодного воздуха в дом сильно его охлаждает. Такие старые окна окна меняют на современные пластиковые которые сохраняют тепло и не пропускают холодный воздух снаружи внутрь дома. В результате обновление свежего воздуха в доме просто останавливается. 

Про проведенным экспериментам из статьи \cite{co2}, следует что:
В помещение площадью 15 $\text{м}^{2}$ (кубатура: 37.5 $\text{м}^3$ - потолок 2.5м), два человека в комнате. 
Проветривание одним окном. Концентрация углекислого газа в этом случае упала с 1765 ppm до 799 ppm (из «красной» зоны в «зелёную») за 12 минут и 17 секунд.
Один человек в комнате и проветривание двумя окнами. 
Проветривание ранним ноябрьским утром дало уменьшение концентрации углекислого газа с 2070 ppm до 763 ppm за 5 минут и 37 секунд.
Следующий эксперимент, прогрев воздуха кондеционером.
В комнате во время замеров было 2 человека. Получилось, что при работающем кондиционере (в режиме "+25, Солнышко") концентрация углекислого газа выросла с комфортных 591 ppm до «красной зоны» в 1200 ppm за 28 минут и 47 секунд.

Кстати, при выключенном кондиционере ситуация похожая. В этой же комнате уровень CO2 растёт с 0,06\% до 0,12\% тоже примерно за полчаса (если в комнате 2 человека). А часа за 2 он доходит до критической отметки 0,3\%, и, по возможности, весьма желательно уже проветривать (хоть на улице и зима :) ).

Как видно из эксперимента при наличии 2 человек в большой 15 метровой комнате необходиом проветривать помещение каждые 2 часа по 15 мин, а лучше чаще.

Если проветривать реже то это очень скажется на самочувствие.  


