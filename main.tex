\documentclass[12pt, twocolumn]{report}

% гипер ссылки
\usepackage{cmap} % чтобы был поиск по pdf

\usepackage[utf8]{inputenc}
\usepackage[english, russian]{babel}
\usepackage[T2A]{fontenc}
\usepackage[unicode]{hyperref}

\usepackage{xcolor}
\usepackage{hyperref}


% русские буквы в формулах
\usepackage{mathtext}
\usepackage{amsmath,amsfonts,amssymb,amsthm,mathtools}
\usepackage{longtable}

\usepackage[left=25mm, right=15mm, top=15mm, bottom=15mm, landscape, a3paper]{geometry}
\tolerance=314

%%%%%%%%%%%%%%%%%%%%%%%%%%%%%%%%%%%%%%%%%%%%%%%%
%%% добавляем точки в Оглавление

\renewcommand{\thechapter}{\arabic{chapter}.}
\renewcommand{\thesection}{\arabic{chapter}.\arabic{section}.}
\renewcommand{\thesubsection}{\arabic{chapter}.\arabic{section}.\arabic{subsection}.}


% отступ при наборе формул в блоке multline
\multlinegap=2cm
% будем считать что переполнения не было, если строка вышла за границу не более чем 0.01pt
\hfuzz=0.01pt
% разрешаю увеличить расстояние между словами не больше чем на 14pt, что бы не было переполнения
\emergencystretch=14pt
% изменение межстрочного интервала
\def\heightline{1.0}
\linespread{\heightline} % 1.3 - это полуторный

\usepackage{hyperref}
%\usepackage[usenames,dvipsnames,svgnames,table,rgb]{xcolor}
\hypersetup{				% Гиперссылки
	unicode=true,           % русские буквы в раздела PDF
	pdftitle={Заголовок},   % Заголовок
	pdfauthor={Автор},      % Автор
	pdfsubject={Тема},      % Тема
	pdfcreator={Создатель}, % Создатель
	pdfproducer={Производитель}, % Производитель
	pdfkeywords={keyword1} {key2} {key3}, % Ключевые слова
	colorlinks=true,       	% false: ссылки в рамках; true: цветные ссылки
	linkcolor=red,          % внутренние ссылки
	citecolor=black,        % на библиографию
	filecolor=magenta,      % на файлы
	urlcolor=cyan           % на URL
}

\title{\textbf{\Huge{Проект одноэтажного загородного дома}}}
\author{ Морозов Андрей }
\date{Нижний Новгород, 2015-2017}


\begin{document}
\maketitle

\tableofcontents

\part{О проекте}

\chapter{Как зарожалось}

\chapter{Проектирование}

% Встал вопрос в чем проектировать и как. Есть огромное количество иструментов и еще большее количество, о которых я не ничего не знаю. 

% И так я выбрал отчет в $\LaTeX$ и рачеты в $Python 3$ + $FreeCAD$ для черчения.

\chapter{Зарубежный опыт}

\newcommand{\kwpm}{$\cfrac{\text{кВт·ч}}{\text{м}^2}\ $}

\section{Пассивные дома}
Пассивный дом, энергосберегающий дом, или экодом (нем. Passivhaus, англ. passive house) -- сооружение, основной особенностью которого является отсутствие необходимости отопления или малое энергопотребление - в среднем около 10\% от удельной энергии на единицу объёма, потребляемой большинством современных зданий. В большинстве развитых стран существуют собственные требования к стандарту пассивного дома.

Растут цены на электричество и тепло. Остро стоит вопрос эксплуатационных затрат на жилье. Показателем энергоэффективности объекта служат потери тепловой энергии с квадратного метра \kwpm в год или в отопительный период. В среднем составляет 100—120 \kwpm. Энергосберегающим считается здание, где этот показатель ниже 40 \kwpm.Для европейских стран этот показатель ещё ниже — порядка 10 \kwpm.

Достигается снижение потребления энергии в первую очередь за счет уменьшения теплопотерь здания.

В Европе существует следующая классификация зданий в зависимости от их уровня энергопотребления:
\begin{itemize}
	\item "Старое здание" (здания построенные до 1970-х годов) — они требуют для своего отопления около трехсот киловатт-часов на квадратный метр в год: 300 \kwpm год.
	\item "Новое здание" (которые строились с 1970-х до 2000 года) — не более 150 \kwpm год.
	\item "Дом низкого потребления энергии» (с 2002 года в Европе не разрешено строительство домов более низкого стандарта) — не более 60 \kwpm год.
	\item "Пассивный дом" — не более 15 \kwpm год.
	\item "Дом нулевой энергии" (здание, архитектурно имеющее тот же стандарт, что и пассивный дом, но инженерно оснащенное таким образом, чтобы потреблять исключительно только ту энергию, которую само и вырабатывает) — 0 \kwpm год.
	\item "Дом плюс энергии" или "активный дом" (здание, которое с помощью установленного на нём инженерного оборудования: солнечных батарей, коллекторов, тепловых насосов, рекуператоров, грунтовых теплообменников и т. п. вырабатывало бы больше энергии, чем само потребляло).
	
\end{itemize}

\section{Энергосберегающие дома}


\section{Экономическая выгода}
В России энергопотребление в домах составляет 400—600 \kwpm в год.

\part{Земельный участок}
\chapter{Нормы и правила расположения объектов на земельном участке}
Согласно СП 53.13330.2011 - ПЛАНИРОВКА И ЗАСТРОЙКА ТЕРРИТОРИЙ САДОВОДЧЕСКИХ (ДАЧНЫХ) ОБЪЕДИНЕНИЙ ГРАЖДАН, ЗДАНИЯ И СООРУЖЕНИЯ, 

Жилое строение или жилой дом должны отстоять от красной линии улиц не менее чем на 5 м, от красной линии проездов — не менее чем на 3 м. При этом между
домами, расположенными на противоположных сторонах проезда, должны быть учтены противопожарные расстояния, указанные в таблице 2. Расстояния от хозяйственных построек до красных линий улиц и проездов должны быть не менее 5 м. По согласованию с правлением садоводческого, дачного объединения навес или гараж для автомобиля может размещаться на участке, непосредственно примыкая к ограде со стороны улицы или проезда.

Минимальные расстояния до границы соседнего участка по санитарно-бытовым условиям должны быть от:
\begin{enumerate}
	\item жилого строения (или дома) — 3 м;
	\item постройки для содержания мелкого скота и птицы — 4 м;
	\item других построек — 1 м;
	\item стволов высокорослых деревьев — 4 м, среднерослых — 2 м;
	\item кустарника — 1 м.
\end{enumerate}

Расстояние между жилым строением (или домом), хозяйственными постройками и границей соседнего участка измеряется от цоколя или от стены дома, постройки (при
отсутствии цоколя), если элементы дома и постройки (эркер, крыльцо, навес, свес крыши и др.) выступают не более чем на 50 см от плоскости стены. Если элементы
выступают более чем на 50 см, расстояние измеряется от выступающих частей или от проекции их на землю (консольный навес крыши, элементы второго этажа, расположенные на столбах и др.).

При возведении на садовом, дачном участке хозяйственных построек, располагаемых на расстоянии 1 м от границы соседнего садового, дачного участка, скат
крыши следует ориентировать таким образом, чтобы сток дождевой воды не попал на соседний участок.


Минимальные расстояния между постройками по санитарно-бытовым условиям должны быть, м:
\begin{enumerate}
	\item от жилого строения или жилого дома до душа, бани (сауны), уборной — 8;
	\item от колодца до уборной и компостного устройства — 8.
\end{enumerate}

Указанные расстояния должны соблюдаться между постройками, расположенными на смежных участках.

В случае примыкания хозяйственных построек к жилому строению или жилому дому расстояние до границы с соседним участком измеряется отдельно от каждого объекта блокировки, например:

\begin{enumerate}
	\item дом-гараж (от дома не менее 3 м, от гаража не менее 1 м);
	\item дом-постройка для скота и птицы (от дома не менее 3 м, от постройки для скота и птицы не менее 4 м).
\end{enumerate}

Гаражи для автомобилей могут быть отдельно стоящими, встроенными или пристроенными к садовому, дачному дому и хозяйственным постройкам.

\chapter{План земельного участка}
Вот земля, нужно разместить все постройки в соответствие с гостом.	


\part{Архитектура}
\chapter{План дома}
Что нам стоит дом построить, нарисуем будем жить.


\chapter{Фундамент}

Скорее всего УШП

\chapter{Первый этаж}

Пирог стены:
Пароизоляция Ютафол Н 110.

Окна монтируются в ЭППС, для предотвращения конденсата на окнах и их промерзания.

\chapter{Крыша}

\chapter{Веранда}

\part{Огонь/Отопление}
\chapter{Зачем и сколько нужно тепла дому?}
У нас есть 5-8 к	вт на отопление, что будем делать?

\part{Вода}

\chapter{Сколько нужно воды на человека?}

Расчет систем водоснабжения производится исходя из следующих норм среднесуточного водопотребления на хозяйственно-питьевые нужды:

\begin{enumerate}
	\item при водопользовании из водоразборных колонок, скважин, шахтных колодцев -- 30-50 л/сут на 1 жителя;
	\item при обеспечении внутренним водопроводом и канализацией (без ванн) -- 125-160 л/сут на 1 жителя.
	\item Для полива посадок на приусадебных участках: овощных культур -- 3-15 л/м2 в сутки; плодовых деревьев -- 10-15 л/м2 в сутки.
\end{enumerate}





\part{Электричество}
\chapter{Основные потребители электроэнергии}
У нас есть 15квт, нужно распорядится ими по умному, незабывая закон Ома $$ I = \cfrac{U}{R} $$

\part{Медные трубы}
\chapter{Трубы - этоже так просто}
Будем использовать трубы рихау и все тут.

\part{Воздух}
\chapter{Немного стандарта}
Стандарт говорит, что воздух нам нужен.

Вопрос в том что непонятно сколько. Возможно достаточно проветривать помещение раз в день - перед сном.

С каждым годом стоимость отопления увеличивается и в погоне за экономией мы стараемся сберечь тепло, которе у нас есть в доме, применяя порой очевидные приемы, которые помогут нам в этом, однако забывая о последствиях. В старых СНиПах 80-х годов было указано, что окна в домах \textbf{должны} обеспечить определенный приток воздуха в помещение. Было дано вполне определенное число по стандарту, сколько иммено воздуха окно должно пропускать. В свою очередь, поступление холодного воздуха в дом сильно его охлаждает. Такие старые окна окна меняют на современные пластиковые которые сохраняют тепло и не пропускают холодный воздух снаружи внутрь дома. В результате обновление свежего воздуха в доме просто останавливается. 

Про проведенным экспериментам из статьи \cite{co2}, следует что:
В помещение площадью 15 $\text{м}^{2}$ (кубатура: 37.5 $\text{м}^3$ - потолок 2.5м), два человека в комнате. 
Проветривание одним окном. Концентрация углекислого газа в этом случае упала с 1765 ppm до 799 ppm (из «красной» зоны в «зелёную») за 12 минут и 17 секунд.
Один человек в комнате и проветривание двумя окнами. 
Проветривание ранним ноябрьским утром дало уменьшение концентрации углекислого газа с 2070 ppm до 763 ppm за 5 минут и 37 секунд.
Следующий эксперимент, прогрев воздуха кондеционером.
В комнате во время замеров было 2 человека. Получилось, что при работающем кондиционере (в режиме "+25, Солнышко") концентрация углекислого газа выросла с комфортных 591 ppm до «красной зоны» в 1200 ppm за 28 минут и 47 секунд.

Кстати, при выключенном кондиционере ситуация похожая. В этой же комнате уровень CO2 растёт с 0,06\% до 0,12\% тоже примерно за полчаса (если в комнате 2 человека). А часа за 2 он доходит до критической отметки 0,3\%, и, по возможности, весьма желательно уже проветривать (хоть на улице и зима :) ).

Как видно из эксперимента при наличии 2 человек в большой 15 метровой комнате необходиом проветривать помещение каждые 2 часа по 15 мин, а лучше чаще.

Если проветривать реже то это очень скажется на самочувствие.  

\chapter{Нормы воздухообмена}

При рассчетах обычно опираются на 30 $\text{м}^3$ на человека в час или 1 объем помещения в час.

Как такие параметры достичь. Вариантов несколько:

\begin{itemize}
	\item Естественная вентиляция. В простейшем случае вытяжные каналы (трубы) из санузлов и кухни на крышу и приток свежего воздуха через форточки. Дешево и сердито.
	\item  Вытяжка принудительная, приток естественный. То есть в вытяжных каналах ставятся вентиляторы. Работают они либо постоянно, либо по датчикам (например, влажности или присутствия в ванной или CO2 в жилых помещениях). Приток опять же либо через форточки, либо через приточные клапаны.
	\item Вытяжка естественная, притокт принудительный.
	То есть в спальнях ставятся системы типа Бризер Tion O2(это приточная вентиляция для квартиры или офиса), а вытяжка через вытяжные канали в сан узлах.
	\item Полноценная приточно-вытяжная система с подачей воздуха в жилые помещения и его забором из нежилых. Переток воздуха в доме осуществляется через щели под дверьми или же через переточные решетки. Часто используется с рекуператором.
\end{itemize}. 


\part{Расчеты}
\chapter{Нагрузка}

\chapter{Сколько стоит дом построить?}
У нас есть 1 млн руб, я могу себе ни в чем не отказывать ;)

Тут нужно попробовать заиспользовать python.



\part{В заключение}

\chapter{Выбор и выводы}

\part{Стандарты}

\chapter{СП и СНиП}

\begin{tabular}{rp{14cm}}
ПУЭ 7 издание    & ПРАВИЛА УСТРОЙСТВА ЭЛЕКТРОУСТАНОВОК \\
СП 7.13130.2013  & Отопление, вентиляция и кондиционирование. Пожарная безопастность \\
СП 14.13330.2011 & Строительство в сейсмических районах \\
СП 15.13330.2012 & Каменные и армокаменные конструкции \\
СП 17.13330.2011 & Кровля \\
СП 20.13330.2011 & Нагрузки и воздействия \\
СП 22.13330.2011 & Основания зданий и сооружений \\
СП 23.101.2004   & Проектирование тепловой защиты здания \\
СП 24.13330.2011 & Свайные фундаменты \\
СП 27.13330.2011 & Бетонные и железобетонные конструкции для высоких температур \\
СП 29.13330.2011 & Полы \\
СП 30.13330.2012 & Внутренний водопровод и канализация зданий \\
СП 31.105.2002   & Проектирование и строительство энергоэффективных одноквартирных жилых домов с деревянным каркасом \\
СП 31.106.2002   & Проектирование и строительство инженерных систем одноквартирных жилых домов \\
СП 31.13330.2012 & Водоснабжение. Наружные сети и сооружения \\
СП 32.13330.2012 & Канализация. Наружные сети с сооружения \\
СП 33.13330.2012 & Расчет на прочность стальных трубопроводов \\
СП 34.13330.2012 & Автомобильные дороги \\
СП 41.103.2000   & Проектирование тепловой изоляции оборудования и трубопроводов \\
СП 41.104.2000   & Проектирование автономных источников теплоснабжения \\
СП 48.13330.2011 & Организация строительства \\
СП 50.13330.2012 & Тепловая защита зданий \\
\end{tabular}

\begin{tabular}{rp{14cm}}
СП 51.13330.2011 & Защита от шума \\
СП 52.13330.2011 & Естественное и искусственное освещение \\
СП 53.13330.2011 & Планировка и застройка территорий дачных зданий \\
СП 54.13330.2011 & Здания жилые и многоквартирные \\
СП 55.13330.2011 & Дома жилые одноквартирные \\
СП 56.13330.2011 & Производственные здания \\
СП 60.13330.2012 & Отопление, вентиляция и кондиционирование воздуха \\
СП 61.13330.2012 & Тепловая изоляция оборудования и трубопроводов \\
СП 63.13330.2012 & Бетонные и железобетонные конструкции \\
СП 64.13330.2011 & Деревянные конструкции \\
СП 70.13330.2012 & Несущие и ограждающие конструкции \\
СП 73.13330.2012 & Внутренние санитарно-технические системы зданий \\
СП 78.13330.2012 & Автомобильные дороги \\
СП 105.13330.2012 & Здания и помещения для хранения и переработки сельскохозяйственной продукции \\
СП 106.13330.2012 & Животноводческие, птицеводческие и звереводческие здания \\
СП 107.13330.2012 & Теплицы и парники \\
СП 113.13330.2012 & Стоянки автомобилей \\
СП 124.13330.2012 & Тепловые сети \\
СП 126.13330.2012 & Геодезические работы в строительстве \\
СП 131.13330.2012 & Строительная климатология \\
СП 132 & Антитеррорестическая защита зданий \\
\end{tabular}

\newpage
\begin{thebibliography}{99}
\addcontentsline{toc}{section}{\refname}
\bibitem{sn}  { СП номер все }
\bibitem{co2} {\small Датчик CO2 — прибор, который подскажет когда проветрить, чтобы думалось эффективнее } \newline \url{https://geektimes.ru/company/dadget/blog/268230}
\bibitem{pro100dom}  {\small ПроСТО Дом }  \newline \url{http://pro100dom.org/}
\bibitem{newshouse}  {\small NewsHouse }   \newline \url{http://www.newshouse.ru}
\bibitem{forumhouse} {\small ForumHouse } \newline \url{http://www.forumhouse.tv}
\end{thebibliography}


%\newpage
%\addcontentsline{toc}{section}{Список иллюстраций}
%\listoffigures

%\newpage
%\addcontentsline{toc}{section}{Список таблиц}
%\listoftables

\end{document}
